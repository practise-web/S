% ---------- INTRODUCTION ----------------
\hiddenchapter{{\large{I}NTRODUCTION}}
\ChapterTitle[\thechapter]{INTRODUCTION}

\hiddensection{{\large{M}}OTIVATION}
\SectionTitle{\thesection}{MOTIVATION}
\\

In the current academic landscape, students and researchers are overwhelmed by the sheer volume of information available online. Finding relevant, credible, and high-quality scholarly resources for specific research questions can be time-consuming and often inefficient. Traditional search engines provide results based on keyword matching rather than contextual understanding, leading to fragmented information and wasted effort.
\\\\
Moreover, existing tools often lack an integrated approach to manage, query, and interact with research documents efficiently. Researchers are forced to manually sift through PDFs, notes, and articles, which can slow down the research process and reduce productivity.
\\\\
\textit{Scholar Mind} is motivated by the need to streamline this process and empower students and researchers with an intelligent, context-aware system. By leveraging advanced natural language processing (NLP) and AI techniques, Scholar Mind aims to understand users' queries, retrieve the most relevant documents, and provide concise, meaningful insights. This approach not only saves time but also enhances the quality of research outcomes, enabling learners to focus on analysis and critical thinking rather than information gathering.
\\\\
Ultimately, the motivation behind \textit{Scholar Mind} is to bridge the gap between knowledge accessibility and knowledge understanding, creating a smarter, faster, and more intuitive way to navigate academic resources.
\\

\hiddensection{{\large{P}}ROBLEM STATEMENT}
\SectionTitle{\thesection}{PROBLEM STATEMENT}
\\

In the era of information overload, students and researchers face significant challenges in efficiently locating and utilizing high-quality academic resources. Existing search engines primarily rely on keyword matching and fail to fully understand the context of a research question, resulting in irrelevant or fragmented results. This forces users to manually sift through large volumes of documents, leading to wasted time and reduced research productivity.
\\\\
Additionally, there is a lack of integrated tools that can both retrieve relevant documents and provide meaningful insights or summaries in a context-aware manner. As a result, researchers struggle to efficiently manage, query, and extract actionable knowledge from available academic resources.
\\\\
\textit{Scholar Mind} addresses this problem by leveraging advanced AI and natural language processing techniques to understand user queries, retrieve relevant documents, and provide concise, context-aware insights, thereby bridging the gap between information availability and knowledge understanding.

\clearpage